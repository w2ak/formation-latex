\documentclass[a4paper, 11pt]{article}

\usepackage[utf8]{inputenc}
\usepackage[T1]{fontenc}

\usepackage[french,noconfigs]{babel}
\usepackage{lmodern}

\begin{document}

  \section{Caractères réservés}

    \{ \} \% \# \$ \^{} \textasciicircum \~{} \textasciitilde \& \_ \textbackslash

    \copyright \dag \ddag \dots \o \O \P \pounds \S \ss \textbar \textperiodcentered \textregistered \texttrademark \textvisiblespace 

    ?` ou \textquestiondown  ou ¿ , !` ou \textexclamdown  ou ¡ , - , -- , --- , << ou « , >> ou »

  \section{Caractères propres au français}

    \og~ceci est une citation~\fg

    \degres{}  \ier{}  \iere{}  \iers{}  \ieme{}  \iemes{}

  \section{Espacements}

    Espace insécable~: à mettre avant certains signes de ponctuation. \smallskip

    Espace de longueur donnée~:\hspace{2cm}peu utile au début. \medskip

    Sauts de ligne… \bigskip

    …de taille variable.

    Un simple retour à la ligne
    ne change pas de paragraphe.

    Pour changer de paragraphe il faut une ligne blanche.

  \section{Commentaires}

    Dans votre code source, ce qui est après un \% n'apparaît pas%, comme vous pouvez le voir.
    . Dès le retour à la ligne, vous n'êtes plus en mode commentaire.


\end{document}
