\documentclass[a4paper, 11pt]{article}
\usepackage[utf8]{inputenc}
\usepackage[T1]{fontenc}
\usepackage[french,noconfigs]{babel}
\usepackage{lmodern}
\usepackage{amsmath,amsfonts,mathrsfs,amssymb}

\title{Créer ses commandes personnalisées avec LaTeX}

\begin{document}
\maketitle

    \section{Un peu de syntaxe}

        \LaTeX est intelligent. \LaTeX\ est intelligent. \LaTeX{} est intelligent.

        $\frac{a}{b}$ $\frac1b$ $\frac{a}2$ $\frac12$ $\frac{12}3$ $\frac\epsilon2$ $\frac1\epsilon$ $\frac\delta\epsilon$

    \section{Créer des commandes personnalisées}

        \subsection{Version simple}

            \newcommand{\ssi}{si et seulement si }
            Je peux utiliser une commande \ssi elle est déjà définie.

        \subsection{Commande avec arguments}

            \newcommand{\suite}[3]{\left({#1}_{#2}\right)_{#2#3}}
            La suite $\suite{u}{n}{\geq0}$ est croissante.

            \newcommand{\comm}[1]{}
            Ceci \comm{n'apparaitra pas dans le texte car c'}est un commentaire.

        \subsection{Argument optionnel}

            \newcommand{\norme}[2][{}]{\left\lVert #2 \right\rVert_{#1}}
            On sait que $\norme{u_n}\rightarrow 0$ tandis que $\norme[\infty]{f_n}\rightarrow\pi$.

    \section{Exercice}

        Créer la commande \verb!\bonjour! telle que :

        \verb!\bonjour{Clément}{Judo}! écrive : \textit{Bonjour, je suis Clément, je suis en section Judo de la promotion 2014.}

        Mais il faut que l'on puisse utiliser la même commande pour écrire : \textit{Bonjour, je suis Denis, je suis en section Escrime de la promotion 2013.}

        \clearpage
        \subsection{Solution}

        \begin{verbatim}
\newcommand{\bonjour}[3][4]{
    {\itshape
        Bonjour, je suis #2, je suis en section #3
        de la promotion 201#1.
    }
}
        \end{verbatim}

        \newcommand{\bonjour}[3][4]{{\itshape Bonjour, je suis #2, je suis en section #3 de la promotion 201#1.}}

        \verb!\bonjour{Clément}{Judo}!~:\\ \bonjour{Clément}{Judo}

        \verb!\bonjour[3]{Denis}{Escrime}!~:\\ \bonjour[3]{Denis}{Escrime}

\end{document}
