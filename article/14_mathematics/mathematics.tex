\documentclass[a4paper, 11pt]{article}
\usepackage[utf8]{inputenc}
\usepackage[T1]{fontenc}
\usepackage{lmodern}
\usepackage[french,noconfigs]{babel}

\usepackage{amsmath,amssymb} % base
\usepackage{amsfonts,mathrsfs} % polices de texte

\begin{document}

\section{La base des équations}

On peut écrire une formule dans un paragraphe : $2+3=5$. Mais on peut aussi (on le fera souvent) l'écrire à part :

\[ 2+3 = 5 \]

Il y a aussi moyen d'avoir des équations numérotées :

\begin{equation}
\label{monequation} % facultatif
f:x\in E\mapsto 3x + 2
\end{equation}

Il y a une foultitude de commandes qui permettent d'afficher des symboles. Certaines commandes affichent des opérateurs : $\mapsto$, $\leq$, $\in$, $\subset$, etc. D'autres sont des opérateurs de taille variable :
\[
  \sum_{i=1}^n
  {
    \frac{1}{n}
  }
\]
Certaines commandes permettent aussi d'écrire les fonctions standard : $\sin(x)$ contre $sin(x)$.

Avec la bibliothèque de commandes fournie par \texttt{amsmath} on peut déjà en écrire beaucoup :
\[
  \forall\varepsilon>0, \exists\delta>0, \forall x,y\in E, |x-y|\leq\delta \Rightarrow |f(x)-f(y)|\leq\varepsilon
\]

Le mieux (au début) est de lire la documentation pour s'y habituer.

\section{Autres éléments}

  \subsection{Polices}

  Différentes polices pour varier :
  \[ x \in \mathbb{C}^* \]
  \[ u \in \mathcal{L}(E,F) \]
  \[ f \in \mathscr{C}^1(\mathbb{R}) \qquad A \in \mathscr{M}_n(\mathbb{K}) \]
  \[ \sigma \in \mathfrak{S}_n \]

  \subsection{Références}

  Les numéros ne servent pas à rien dans les équations : vous pouvez ensuite parler de la formule \eqref{monequation} qui se trouve page \pageref{monequation}

  \subsection{Parenthésages}

  On signale à \LaTeX{} quelles sont les parenthèses gauche et droite si l'on veut qu'il adapte la taille :
  \[
  \left[
    0,
    \left(
      \frac{n}{n+1}
    \right)
  \right[
  \]
  contre
  \[
  [0,(\frac{n}{n+1})[
  \]

  \subsection{Etc.}

  De nombreuses choses sont disponibles (systèmes, matrices, intégrales, etc.) et détaillées dans les documentations.

\end{document}
