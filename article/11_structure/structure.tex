\documentclass[a4paper, 11pt]{article}

\usepackage[utf8]{inputenc}
\usepackage[T1]{fontenc}

\usepackage[french,noconfigs]{babel}
\usepackage{lmodern}

\begin{document}

\setcounter{tocdepth}{1}%5
\tableofcontents

\clearpage

  \section{Structure du document} \label{structure}

    \subsection{Sous-section}

      \subsubsection{Sous-sous-section}

        \paragraph{Lorem ipsum} dolor sit amet, consectetur adipiscing elit. In pharetra pellentesque massa ut mattis. Curabitur dignissim lacus orci, ornare faucibus orci euismod vitae. Donec viverra risus nisl, ac pellentesque mauris efficitur id. Nulla sed nisl eu mi ultricies eleifend quis commodo est.

        Sed ornare dignissim ex, sit amet sodales ipsum finibus id. Vestibulum ante ipsum primis in faucibus orci luctus et ultrices posuere cubilia Curae; Vivamus pharetra fringilla dui, vel hendrerit odio posuere id. Donec efficitur, ex vel consectetur aliquam, odio eros feugiat ligula, non iaculis orci lorem id ipsum. Phasellus ut pulvinar libero. Donec vel nulla at velit dignissim ornare eget eu magna. Duis a ligula ante. Sed eu ultricies tortor. Aliquam erat volutpat.

      \subsubsection*{Sous-sous-section non numérotée}

        \subparagraph{Donec ullamcorper} scelerisque elit, a pellentesque erat eleifend nec. In et nisi mi. Nam at mattis arcu. Nulla malesuada, purus eget sagittis ultricies, nisl ligula tristique quam, at hendrerit felis nisl et velit. Vivamus aliquam ornare turpis sed sollicitudin. Integer fringilla enim at leo sodales, vitae egestas leo mollis. Donec vestibulum consectetur tortor, eget commodo libero faucibus ac. In dignissim risus odio, sit amet ornare est malesuada a. Donec scelerisque quam enim, a viverra enim mollis at. Sed in blandit ligula. 

\clearpage
  \section{Listes}

    \subsection{Numérotée}

    \begin{enumerate}
      \item Lorem ipsum
      \item dolor sit amet,
      \item consectetur adipiscing elit.
    \end{enumerate}

    \subsection{Non-numérotée}

    \begin{itemize}
      \item Lorem ipsum.
      
      etsuanb c
      etusrn
      \item dolor sit amet,
      \item consectetur adipiscing elit.
    \end{itemize}

    \subsection{Description}

    \begin{description}
      \item[Lorem] ipsum
      \item[dolor] sit amet,
      \item[consectetur] adipiscing elit.
    \end{description}

    \subsection{Imbrications}\label{listesimbriquees}

    \begin{enumerate}
      \item Lorem
      \begin{itemize}
        \item ipsum
        \item dolor
        \begin{enumerate}
          \item sit
          \item amet
        \end{enumerate}
        \item consectetur
      \end{itemize}
      \item adipiscing
    \end{enumerate}

\clearpage
  \section{Références}

    \subsection{Notes de bas de page}

    Lorem ipsum dolor sit amet, consectetur\footnote{In pharetra pellentesque massa ut mattis.} adipiscing elit. Curabitur dignissim lacus orci, ornare faucibus orci euismod vitae.

    \subsection{Références croisées}

    On a parlé de la structure du document dans la section \ref{structure} page \pageref{structure}, et des listes imbriquées en section \ref{listesimbriquees} page \pageref{listesimbriquees}.

\clearpage
  \section{Mise en forme du texte}

    \subsection{Style}

      \paragraph{Italique. } On peut mettre du texte \textit{en italique}, ou aussi tout un paragraphe :

      {\itshape Lorem ipsum dolor sit amet, consectetur adipiscing elit. In pharetra pellentesque massa ut mattis. Curabitur dignissim lacus orci, ornare faucibus orci euismod vitae.}

      \paragraph{Gras. } On peut mettre du texte \textbf{en gras}, ou aussi tout un paragraphe :

      {\bfseries Lorem ipsum dolor sit amet, consectetur adipiscing elit. In pharetra pellentesque massa ut mattis. Curabitur dignissim lacus orci, ornare faucibus orci euismod vitae.}

      \paragraph{Petites majuscules. } On peut mettre du texte \textsc{en petites Majuscules}, ou aussi tout un paragraphe :

      {\scshape Lorem ipsum dolor sit amet, consectetur adipiscing elit. In pharetra pellentesque massa ut mattis. Curabitur dignissim lacus orci, ornare faucibus orci euismod vitae.}

      \paragraph{Emphase. } L'emphase sert à \emph{accentuer} du texte. {\itshape Elle s'adapte au \emph{style} de l'endroit où l'on est.} {\bfseries C'est pratique d'avoir une seule commande qui \emph{accentue} dans toutes les situations.}

      \paragraph{Miscellaneous. } On peut \underline{souligner} du texte, ou le \textsuperscript{mettre en exposant}. On peut aussi écrire \oldstylenums{314159365} avec un style ancien, comparé à 314159265.

    \subsection{Fonte}

      \paragraph{Fonte par défaut. } La \textnormal{fonte par défaut} peut être changée si nécessaire. {\normalfont Il y a des commandes pour écrire avec la fonte par défaut. Ça permet par exemple d'y revenir si on a écrit autrement.}

      \paragraph{Serif. } La fonte par défaut est \textrm{serif} si on ne l'a pas changée. {\rmfamily C'est une fonte avec \emph{empattements}.}

      \paragraph{Sans serif. } On peut utiliser la fonte \textsf{sans serif}. {\sffamily C'est une fonte \emph{sans empattements}.}

      \paragraph{Machine à écrire. } On peut utiliser la fonte \texttt{teletype}. {\ttfamily C'est une fonte \emph{à largeur fixe}.}

    \subsection{Corps}

      On peut \large changer la \Large taille du \LARGE texte que l'on \Huge écrit. \normalsize Attention à \small bien remettre \footnotesize la taille \scriptsize d'origine de votre texte. \normalsize

    \subsection{Alignement}

      Par défaut les paragraphes sont justifiés. On peut aussi les aligner à droite, gauche ou au centre.

      Donec ullamcorper scelerisque elit, a pellentesque erat eleifend nec. In et nisi mi. Nam at mattis arcu. Nulla malesuada, purus eget sagittis ultricies, nisl ligula tristique quam, at hendrerit felis nisl et velit. Vivamus aliquam ornare turpis sed sollicitudin.

      \begin{flushleft}
      Donec ullamcorper scelerisque elit.
      \end{flushleft}

      \begin{center}
      Donec ullamcorper scelerisque elit.
      \end{center}

      \begin{flushright}
      Donec ullamcorper scelerisque elit.
      \end{flushright}

  \section{Table des matières}

    La table des matières s'obtient avec \texttt{\textbackslash tableofcontents}. On peut changer sa \emph{profondeur} avec {\ttfamily \textbackslash setcounter\{tocdepth\}\{{\normalfont\itshape (nombre)}\}}.


\end{document}
