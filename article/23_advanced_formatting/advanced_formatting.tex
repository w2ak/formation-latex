\documentclass[a4paper, 11pt]{article}

\usepackage[utf8]{inputenc}
\usepackage[T1]{fontenc}

\usepackage[french,noconfigs]{babel}
\usepackage{lmodern}

\usepackage{xcolor}

\begin{document}

  \section{La césure}

        \subsection{Qu'est-ce que c'est~?}

            Les habitants de la ville de Niederschaeffolsheim, sont les Niederschaeffolsheimois.

        \subsection{Aider la césure}

            Quand LaTeX fait mal la césure, on peut lui indiquer comment anti\-consti\-tu\-tion\-nel\-le\-ment se coupe. On peut aussi en début de document préciser les mots difficiles à couper~:

            \begin{center}
                \verb!\hyphenation{!\textit{anti-consti-tu-tion-nel-le-ment}\verb!}!
            \end{center}

        \subsection{Empêcher la césure}

            Pour forcer l'absence de césure\footnote{Par exemple pour un nom propre.}, on peut utiliser la commande \verb!\mbox{…}!~:

            \begin{center}
                \verb!\mbox{!\textit{Vaneau}\verb!}!
            \end{center}

    \section{Les couleurs}

        \begin{center}
            \itshape\bfseries On prendra garde à ne pas abuser de l'usage de couleurs dans les documents. Typiquement, un document scientifique est d'autant plus aprécié qu'il est sobre.
        \end{center}

        On utilisera le package (extension) \fbox{\texttt{xcolor}}~:

        \begin{center}
            \verb!\usepackage{!\textit{xcolor}\verb!}!
        \end{center}

        Certaines couleurs par défaut sont alors rendues disponibles~:

        \renewcommand{\c}[1]{\textcolor{#1}{#1}}
        \newcommand{\bc}[1]{\colorbox{black}{\c{#1}}}

        \begin{center}
            \sffamily\bfseries
            \begin{tabular}{|*{5}{c|}}
                \hline
                \c{red} & \c{green} & \c{blue} & \c{cyan} & \c{magenta} \\
                \bc{red} & \bc{green} & \bc{blue} & \bc{cyan} & \bc{magenta} \\
                \hline
                \c{yellow} & \c{orange} & \c{violet} & \c{purple} & \c{brown} \\
                \bc{yellow} & \bc{orange} & \bc{violet} & \bc{purple} & \bc{brown} \\
                \hline
                \c{black} & \c{darkgray} & \c{gray} & \c{lightgray} & \c{white} \\
                \bc{black} & \bc{darkgray} & \bc{gray} & \bc{lightgray} & \bc{white} \\
                \hline
            \end{tabular}
        \end{center}

\clearpage

        \subsection{Écrire du texte en couleur}

            Pour écrire du texte en couleur on utilise l'une de ces deux syntaxes:

            \begin{center}
                \verb!\textcolor{!\textit{couleur}\verb!}{!\textit{texte}\verb!}! \\
                \verb!{\color{!\textit{couleur}\verb!} !\textit{texte}\verb! }!
            \end{center}

        \subsection{Utiliser ses propres couleurs}

            On peut utiliser des couleurs personnalisées :

            \begin{flushleft}
                \verb!\definecolor{!\textit{bleu303}\verb!}{!\textit{RGB}\verb!}{!\textit{0,62,92}\verb!}!
                \verb!\textcolor{!\textit{bleu303}\verb!}{!\textit{texte}\verb!}!

                \verb!\textcolor{!\textit{red!30!white}\verb!}{!\textit{texte}\verb!}!
                \verb!\textcolor{!\textit{red!70!white}\verb!}{!\textit{texte}\verb!}!

                \verb!\textcolor{!\textit{red!25!green!25!blue}\verb!}{!\textit{texte}\verb!}!
            \end{flushleft}

            Ce qui donne :

            \begin{flushleft}
                \sffamily\bfseries
                \definecolor{bleu303}{RGB}{0,62,92}
                \textcolor{bleu303}{texte}
                
                \textcolor{red!30!white}{texte}
                \textcolor{red!70!white}{texte}

                \textcolor{red!40!blue!40!black}{texte}
            \end{flushleft}

    \section{Les cadres}

        \subsection{Encadrer du texte}

            Sur quelques mots : \fbox{c'est fait ainsi}.

            Sur plusieurs lignes :
            \fbox{
            \begin{minipage}[t]{5cm}
            On peut ici mettre du texte sur plusieurs lignes, et ce sans problèmes.
            \end{minipage}
            }

        \subsection{Un fond coloré}

            On peut utiliser \colorbox{red}{des boites} ou \fcolorbox{blue}{red}{des boîtes encadrées}.

            Et si on est cinglé on peut aussi changer la couleur de toute la page. \pagecolor{lightgray}

\clearpage
\pagecolor{white}

    \section{Interligne}

        Il y a de nombreuses manières de modifier l'interligne. En voici deux assez propre. On peut remplacer \texttt{1.2} par n'importe quel facteur multiplicatif.

        \begin{itemize}
            \item Commande globale et simple : \\
            \verb!\renewcommand{\baselineskip}{1.2\baselineskip}!
            \item Avec le package \fbox{\texttt{setspace}}, commande locale : \\
            \verb!\begin{spacing}{1.2}!\\
            \verb!...!\\
            \verb!\end{spacing}!
        \end{itemize}

    \section{Colonnes}

        \subsection{Deux colonnes}

            On peut avoir un document complet avec deux colonnes :

            \begin{center}
                \verb!\documentclass[!\textit{…, twocolumn}\verb!]{!\textit{article}\verb!}!
            \end{center}

            On peut aussi localement alterner entre une et deux colonnes :

            \begin{flushleft}
                \verb!\twocolumn[!\textit{\textbackslash section\{Titre…\}}\verb!]!

                …

                \verb!\onecolumn!
            \end{flushleft}

        \subsection{Plus de colonnes}

            Avec l'extension \fbox{\texttt{multicol}} on peut utiliser :

            \begin{flushleft}
                \verb!\begin{multicols}{n}!\\ … \\ \verb!\end{multicols}!
            \end{flushleft}

            La commande \verb!\columnbreak! permet de forcer le changement de colonne.
\clearpage
    \section{Utiliser plusieurs langues}

        \begin{verbatim}
    \usepackage[french,english]{babel}
     …
    \begin{document}
         …
        \selectlanguage{english}
         …
        \selectlanguage{french}
         …
    \end{document}
        \end{verbatim}



\end{document}
